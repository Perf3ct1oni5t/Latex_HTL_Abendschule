%-----------------------------------------------------------
%----------------------DIPLOMARBEIT-------------------------
%-----------------------------------------------------------
%
%
%-------------VERWENDUNG-VON-ABKÜRZUNGEN--------------------
% Mit dem Befehl \ac{xxx} referenziert man auf Abkürzungen im Abk-Verzeichnis. 
% Achtung diese Abkürzungen werden beim ersten Mal ausgeschrieben und abgekürzt verwendet.
% Möchte man das nicht, dann muss man den Befehl \acs{xxx} verwenden
%
%-------------DEFINIEREN-VON-ABKÜRZUNGEN-------------------- 
% \acro{Abk.}[Abkürzung im Text]{Voller Name}
% BSP:
% \acro{I2R}[$I^{2}R]{Wärmeverluste}
%-----------------------------------------------------------
%
%
% Falls der Befehl \autoref{} oder \SI{}{} nicht erkannt wird, müssen die Befehle \usepackage{hyperref} und \usepackage{siunitx} in das Root Dokument kopiert werden und dann einmal kompilieren. Danach kann der Befehl wieder gelöscht werden.
%
% Die im Quellenverzeichnis angeführten Quellen werden mit dem Befehl \cite{NAMEDERQUELLE} zitiert.
%
%
%
% mit \responsible{Name des Autors} definiert man den Schüler, der den Text zum thema verfasst hat.
%
%
%
%
%

% -----------------------------------------------------------
% ----------------------AUFZÄHLUNGEN ------------------------
%
% \begin{itemize}
%\item XXX
%\item XXX
%\end{itemize}
%
%-----------------------------------------------------------
% -------------STANDARD TABELLE IN DER MITTE-----------------
%
%\begin{table}[htp]
%   \centering
%    \begin{tabular}{|c|c|c|}
%        \hline
%        \multicolumn{3}{|c|}{{\ul \textbf{Geräteliste}}}                       \\ \hline
%        \textbf{Gerät}         & \textbf{Name}         & \textbf{Nummer}       \\ \hline
%        Voltmeter              & x                     & x                     \\ \hline
%        Amperemeter            & x                     & x                     \\ \hline
%        Wattmeter              & x                     & x                     \\ \hline
%        Oszilloskop            & x                     & x                     \\ \hline
%        Funktionsgenerator     & x                     & x                     \\ \hline
 %       \multicolumn{1}{|l|}{} & \multicolumn{1}{l|}{} & \multicolumn{1}{l|}{} \\ \hline
%    \end{tabular}
%\end{table}
%
%-----------------------------------------------------------
%-----------------EINFÜGEN VON BILDERN----------------------
%
%\begin{figure} [htp]
%    \centering
%    \includegraphics[angle=-90, width=0.8\textwidth]{img/NAME_DES_BILDES}
%    \caption{Ergebnis der Simulation}
%    \label{figure:simulation}
%\end{figure}
%
%
%
% Dieses Bild wird um 90° im Uhrzeigersinn durch den Befehl "angle=-90" gedreht.
%
%
% Am besten oben auf Assistenten drücken und Grafik einfügen klicken
%
%-----------------------------------------------------------
%----------------2 BILDER NEBENEINANDER---------------------
%
%\begin{figure}[htb!]
%    \centering
%    \begin{minipage}[t]{0.45\linewidth}
%        \centering
%        \includegraphics[width=\linewidth]{img/NAME_DES_BILDES}
%        \caption{CAPTION EINFÜGEN}
%        \label{fig:Bild1}
%    \end{minipage}%
%    \hfill
%    \begin{minipage}[t]{0.45\linewidth}
%        \centering
%        \includegraphics[width=\linewidth]{img/NAME_DES_BILDES_2}
%        \caption{CAPTION EINFÜGEN}
%        \label{fig:Bild2}
%    \end{minipage}%      
%\end{figure}
%
%
%----------------------------------------------------------
%--------------------ERKENNTNISBOX-------------------------
%
%\begin{erkenntnisbox}
%%       ....deine Erkenntnis
%%\end{erkenntnisbox}
%
% Zeichnet eine Erkenntnisbox mit schwarzen Rahmen - weils cool is.
%
%
%----------------------------------------------------------
%---------------FORMEL MIT SI EINHEITEN--------------------
% Beispiele: 
%  100mm^2      $$ \SI{100}{\milli \meter ^2}$$
%  50kVA        $$ \SI{50}{\kilo \volt \ampere}$$
%  10 Sm/mm^2   $$ \SI{10}{\dfrac{\siemens \meter}{\milli \meter ^2}}
%  10°C         $$ \SI{10}{\celsius}$$
%
% Es gibt sehr viele Einheiten, einfach ausprobieren :)
%
%
%
% Um Die tiefgestellten Buchstaben gerade darzustellen \mathrm{} verwenden.
% Beispiel:  
% Pges = 1000W  $$ P_\mathrm{gesamt} = \SI{1000}{\watt} $$
%
%
%-----------------------------------------------------------
%---------------------BILDER SPERREN------------------------
% \floatbarrier 
% zum setzen einer Barriere, dass Bilder nicht woanders sind
%
%
%
%-----------------------------------------------------------
%---------------------PDF einfügen--------------------------
%\includepdf[pages=-, scale=1]{../Dateispeicherort/pdf.pdf}
%
% Diese Packages gehören in die Packages.tex
%\usepackage{pdfpages}
%\usepackage{afterpage}
